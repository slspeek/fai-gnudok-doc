\documentclass[12pt,a4paper]{article}
\usepackage[dutch]{babel}
\usepackage[utf8]{inputenc}
\usepackage[margin=0.5in]{geometry}
\usepackage{amsmath}
\usepackage{amsfonts}
\usepackage{amssymb}
\usepackage{graphicx}
\usepackage{listings}
\usepackage{float}
\usepackage{hyperref}


\begin{document}
\graphicspath{ {./images/} }
\DeclareGraphicsExtensions{.png,.jpg}
\lstset{language=bash}      
\author{Steven L. Speek}
\title{FAI GNUDok documentatie}
\date{\today}
\maketitle
\abstract{Nederlandse handleiding voor een FAI-installatie.}

\section{Voorbereiding}
\subsection{MAC adres achterhalen}
Laat de machine die je wilt installeren opstarten en ga de BIOS in. In de BIOS schakel je boot ROM faciliteiten in. En configureer je de netwerkkaart als eerste boot-optie. Dan laat je de netwerkboot lopen en schrijft van de scherm het MAC-adres over. Dit zijn zes groepjes van twee cijfers en letters met dubbele punten ertussen. 

\section{FAI server instellen}
\subsection{Hosts bestand}
Voor de installatie van machine voor de verkoop gebruiken we hostnamen zoals \texttt{debianNN} (dit is bijvoorbeeld \texttt{debian02}).
In \texttt{/etc/hosts} moet een hostnaam en een IP-adres worden gedefinieerd (zie eventueel \texttt{man hosts}). Met het volgende commando bepaal je de hoogst in gebruik zijnde \texttt{debianNN} en ip-adres.  
\begin{lstlisting}
grep debian /etc/hosts|sort -r|head -1
\end{lstlisting}
Je gaat precies \'{e}\'{e}n hoger zitten. Deze voeg je toe aan \texttt{/etc/hosts}, met het commando
\begin{lstlisting}
vim /etc/hosts
\end{lstlisting}
\subsection{DHCP-server instellen}
Met het speciaal daarvoor bestemde fai-util \texttt{dhcp-edit} voeg je deze in vorige sectie gekozen hostnaam en mac-adres toe.
\begin{lstlisting}
dhcp-edit debianNN 00:11:22:33:44:55
\end{lstlisting}
Dan moet de DHCP-server herstart worden met:

\begin{lstlisting}
/etc/init.d/isc-dhcp-server restart
\end{lstlisting}


\subsection{PXE instellen}
Met het volgende commando:
\begin{lstlisting}
pxe.sh debianNN
\end{lstlisting}
maak je een PXE entry voor jouw machine aan.

Nu zou je je te installeren machine kunnen laten booten van haar netwerkkaart en de automatische installatie zou moeten beginnen.

\end{document}

